\section{Generalidades}
\label{sec:generalidades}

\subsection{Problemática}
\label{subsec:problematica}

\subsubsection{Introducción}
\label{subsubsec:introduccion}

Se presenta una breve introducción al tema de la investigación, incluyendo el contexto y la importancia del mismo.

Esta debe servir de motivación tanto para el lector como para el autor, y debe incluir una breve descripción de la problemática
que se va a abordar en el proyecto.

\subsubsection{Árbol de Problemas}
\label{subsubsec:arbol_problemas}

Se describe la necesidad de usar un árbol de problemas, comúnmente citando a \textcite{vesely2008arbol}.

Además, se incluye el esquema de árbol de problemas.

\begin{figure}[h!]
    \centering
    \begin{tikzpicture}[
            node distance=2cm and 0.5cm,
            every node/.style={font=\footnotesize, align=center},
            cause/.style={rectangle, draw=blue, fill=blue!10, rounded corners, text width=3cm},
            effect/.style={rectangle, draw=orange!80!black, fill=orange!20, rounded corners, text width=3cm},
            core/.style={rectangle, draw=black, thick, rounded corners, fill=white, text width=5.5cm, align=center},
            arrow/.style={-Latex, thick}
        ]

        \node[core] (problem) {Problema central};

        \node[effect, above=1.5cm of problem, xshift=-6.5cm] (effect1) {Efecto 1};
        \node[effect, above=1.5cm of problem, xshift=-2cm] (effect2) {Efecto 2};
        \node[effect, above=1.5cm of problem, xshift=2cm] (effect3) {Efecto 3};
        \node[effect, above=1.5cm of problem, xshift=6.5cm] (effect4) {Efecto 4};

        \node[cause, below=1.5cm of problem, xshift=-6.5cm] (cause1) {Causa 1};
        \node[cause, below=1.5cm of problem, xshift=-2cm] (cause2) {Causa 2};
        \node[cause, below=1.5cm of problem, xshift=2cm] (cause3) {Causa 3};
        \node[cause, below=1.5cm of problem, xshift=6.5cm] (cause4) {Causa 4};

        \draw[arrow, orange!80!black] (problem) -- (effect1);
        \draw[arrow, orange!80!black] (problem) -- (effect2);
        \draw[arrow, orange!80!black] (problem) -- (effect3);
        \draw[arrow, orange!80!black] (problem) -- (effect4);

        \draw[arrow, blue] (cause1) -- (problem);
        \draw[arrow, blue] (cause2) -- (problem);
        \draw[arrow, blue] (cause3) -- (problem);
        \draw[arrow, blue] (cause4) -- (problem);

    \end{tikzpicture}
    \caption{Árbol de Problemas}
    \label{fig:arbol_problemas}
\end{figure}

\subsubsection{Descripción}
\label{subsubsec:descripcion}

Se describe de forma narrada el árbol de problemas, explicando las causas y su relación con los efectos,
así como con la problemática central.

\subsubsection{Problema Seleccionado}

Se detalla de manera concreta el problema seleccionado para la investigación, explicando su relevancia y
justificando su elección.

\subsection{Objetivos}
\label{subsec:objetivos}

En esta sección se presentan el objetivo general y los objetivos específicos del proyecto. El objetivo general
se enfoca en la resolución de la problemática seleccionada, mientras que los objetivos específicos están relacionados con
los problemas-causa planteados en el árbol de problemas.

\subsubsection{Objetivo General}
\label{subsubsec:objetivo_general}

Se define el objetivo general de la investigación.

\subsubsection{Objetivos Específicos}
\label{subsubsec:objetivos_especificos}

Se detallan los objetivos específicos de la investigación, que deben ser medibles y alcanzables.

\textbf{O1. <Verbo en infinitivo>} <Descripción del objetivo específico 1>.

\textbf{O2. <Verbo en infinitivo>} <Descripción del objetivo específico 2>.

\textbf{O3. <Verbo en infinitivo>} <Descripción del objetivo específico 3>.

\textbf{O4. <Verbo en infinitivo>} <Descripción del objetivo específico 4>.

\subsubsection{Resultados Esperados}
\label{subsubsec:resultados_esperados}

Se explican los resultados esperados de acuerdo a los objetivos específicos planteados en la sección anterior.

\subsubsection{Mapeo de Resultados Esperados y Medios de Verificación}
\label{subsubsec:mapeo_objetivos}

Esta sección detalla cómo cada resultado esperado de la investigacion se asocia a un medio de verificación específico, y a indicadores
objetivamente verificables (IOV). Este mapeo permite establecer una relación clara entre los objetivos planteados y los métodos
empleados para evaluar su cumplimiento, facilitando así la medición del progreso y el éxito del proyecto.

\begin{table}[h!]
    \centering
    {\footnotesize
        \begin{tabular}{|p{4cm}|p{5.5cm}|p{5.5cm}|}
            \hline
            \multicolumn{3}{|p{15cm}|}{\textbf{O1. Objetivo específico 1.}}                                             \\
            \hline
            \textbf{Resultado Esperado} & \textbf{Medio de Verificación}           & \textbf{IOV}                       \\
            \hline
            R1. Resultado Esperado 1    & Forma de Verificar el resultado esperado & Indicador objetivamente verifiable \\
            \hline
            R2. Resultado Esperado 2    & Forma de Verificar el resultado esperado & Indicador objetivamente verifiable \\
            \hline
            R3. Resultado Esperado 3    & Forma de Verificar el resultado esperado & Indicador objetivamente verifiable \\
            \hline
        \end{tabular}
    }
    \caption{Resultados Esperados, Medios de Verificación e IOV para el Objetivo 1}
    \label{tab:mapeo_objetivos_1}

\end{table}

Lo mismo se hace para los demás objetivos específicos.