\section{Estado del Arte}
\label{sec:sota}

\subsection{Introducción}
\label{subsec:sota-intro}

En esta sección, se explica el proceso de revisión sistemática de la literatura, empleado para
analizar de manera rigurosa y completa la evidencia existente sobre el tema propuesto.

Comúnmente se cita a \textcite{kitchenham2004procedures} para ello.

\subsection{Objetivos de la Revisión}
\label{subsec:objetivos-revision}

Se explica el objetivo principal de esta revisión sistemática, y se identifican objetivos específicos que
lo respalden.

\begin{itemize}
      \item Explicación del objetivo específico 1 de la revisión sistemática.
      \item Explicación del objetivo específico 2 de la revisión sistemática.
      \item Explicación del objetivo específico 3 de la revisión sistemática.
      \item Explicación del objetivo específico 4 de la revisión sistemática.
      \item Explicación del objetivo específico 5 de la revisión sistemática.
\end{itemize}

\subsection{Preguntas de Revisión}
\label{subsec:preguntas-revision}

Se plantearon preguntas de revisión de manera que aborden temas relevantes en la investigación y
se mapeen directamente con los objetivos específicos antes mencionados.

\begin{itemize}
      \item \textbf{P1}: Pregunta de revisión 1 relacionada con el objetivo específico 1.
      \item \textbf{P2}: Pregunta de revisión 2 relacionada con el objetivo específico 2.
      \item \textbf{P3}: Pregunta de revisión 3 relacionada con el objetivo específico 3.
      \item \textbf{P4}: Pregunta de revisión 4 relacionada con el objetivo específico 4.
      \item \textbf{P5}: Pregunta de revisión 5 relacionada con el objetivo específico 5.
\end{itemize}

\subsection{Protocolo de Búsqueda}
\label{subsec:protocolo-busqueda}

\subsubsection{Motores de Búsqueda}
\label{subsubsec:motor-busqueda}

Se muestran los mosotores de búsqueda usados y su justificación.

\subsubsection{Cadenas de Búsqueda}
\label{subsubsec:cadenas-busqueda}

Se definen las cadenas de búsqueda y se presenta la cantidad de resultados obtenidos
por motor de búsqueda.

\begin{table}[h!]
      \centering
      \begin{tabular}{|p{2cm}|p{10cm}|p{2.5cm}|}
            \hline
            \textbf{Motor de búsqueda} & \textbf{Cadena de búsqueda}                   & \textbf{Número de Resultados}                   \\ \hline
            \textbf{Motor 1}           & \texttt{(Cadena de búsqueda para el motor 1)} & Cantidad de artículos encontrados en el motor 1 \\ \hline
            \textbf{Motor 2}           & \texttt{(Cadena de búsqueda para el motor 2)} & Cantidad de artículos encontrados en el motor 2 \\ \hline
            \textbf{Motor 3}           & \texttt{(Cadena de búsqueda para el motor 3)} & Cantidad de artículos encontrados en el motor 3 \\ \hline
      \end{tabular}
      \caption{Cadenas y resultados en diferentes motores de búsqueda.}
      \label{tab:search-results}
\end{table}

\subsubsection{Criterios de Inclusión y Exclusión}
\label{subsubsec:criterios-inclusion-exclusion}

Se definen los criterios de inclusión y exclusión para garantizar que los estudios seleccionados sean relevantes y
cumplan con los objetivos de la revisión sistemática.

\subsubsection{Documentos Seleccionados}
\label{subsubsec:documentos-seleccionados}

Tras la aplicación de los criterios de inclusión y exclusión, se muestran los documentos seleccionados.

\begin{table}[h!]
      \centering
      {\scriptsize
            \begin{tabular}{|p{1cm}|p{13cm}|}
                  \hline
                  \textbf{ID} & \textbf{Referencia}                   \\ \hline
                  SR001       & Cita en formato APA del documento 1.  \\ \hline
                  SR002       & Cita en formato APA del documento 2.  \\ \hline
                  SR003       & Cita en formato APA del documento 3.  \\ \hline
                  SR004       & Cita en formato APA del documento 4.  \\ \hline
                  SR005       & Cita en formato APA del documento 5.  \\ \hline
                  SR006       & Cita en formato APA del documento 6.  \\ \hline
                  SR007       & Cita en formato APA del documento 7.  \\ \hline
                  SR008       & Cita en formato APA del documento 8.  \\ \hline
                  SR009       & Cita en formato APA del documento 9.  \\ \hline
                  SR010       & Cita en formato APA del documento 10. \\ \hline
            \end{tabular}
      }
      \caption{Documentos seleccionados para la revisión sistemática.}
      \label{tab:documentos-seleccionados}
\end{table}

\clearpage
\subsection{Formulario de Extracción}
\label{subsec:formulario-extraccion}

A continuación, se presenta el formulario de extracción que servirá para recopilar la información que se considere
relevante y para responder a las preguntas de revisión.

\begin{table}[h!]
      \centering
      {\footnotesize
            \begin{tabular}{|p{4cm}|p{9cm}|l|}
                  \hline
                  \textbf{Campo}         & \textbf{Descripción}                          & \textbf{Pregunta} \\ \hline
                  ID                     & Identificador primario del estudio            & General           \\ \hline
                  Título                 & Título del estudio                            & General           \\ \hline
                  Autores                & Autores que formaron parte del estudio        & General           \\ \hline
                  Año                    & Año de publicación                            & General           \\ \hline
                  Fuente                 & Nombre de la revista o conferencia            & General           \\ \hline
                  Enlace de consulta     & URL al artículo completo                      & General           \\ \hline
                  Abstract               & Resumen del estudio                           & General           \\ \hline
                  Citaciones             & Número de citas recibidas                     & General           \\ \hline
                  Campo relacionado a P1 & Campo relacionado a la pregunta 1 de revisión & P1                \\ \hline
                  Campo relacionado a P2 & Campo relacionado a la pregunta 2 de revisión & P2                \\ \hline
                  Campo relacionado a P3 & Campo relacionado a la pregunta 3 de revisión & P3                \\ \hline
                  Campo relacionado a P4 & Campo relacionado a la pregunta 4 de revisión & P4                \\ \hline
                  Campo relacionado a P5 & Campo relacionado a la pregunta 5 de revisión & P5                \\ \hline
            \end{tabular}
      }
      \caption{Formulario de extracción de datos para la revisión sistemática.}
      \label{tab:formulario-extraccion}
\end{table}

\subsection{Resultados de la Revisión}
\label{subsec:resultados-revision}

Se responde a cada una de las preguntas de revisión planteadas.